\documentclass[12pt]{article}
\usepackage[utf8]{inputenc}
\usepackage{ctex} % 支持中文
\usepackage{amsmath} % 数学公式
\usepackage{graphicx} % 插入图片
\usepackage{geometry} % 页面布局
\geometry{a4paper,scale=0.8}

\title{计算器程序设计报告}
\author{朱孔阳}

\begin{document}

\maketitle

\section{写在前面}
如果你看到这段话,意味着我没能修复前一版的程序,那版是用了中缀转前缀写的,但是输出变成了这个:������һ����׺���ʽ1+1��������2。QwQ。所以这一版干脆用了个比较符合人类阅读习惯的写法。
\section{PS}
老一版分开写的,保存在了`prev`文件夹里,但是没有`main.cpp`。

\section{符合人类宝宝体质的设计思路}
\subsection{词法分析}
首先检查每个字符是否合法,包括数字、运算符、括号和空格。非法字符将导致程序异常退出。

\subsection{解析表达式}
用两个栈来存储操作数和操作符。操作数栈用于存储数字,而操作符栈用于存储运算符和括号。函数 `precedence` 来确定运算符的优先级。加减、乘除优先级为1、2。

\subsection{应用运算符}
 `applyOp` 函数来执行具体的运算操作。程序会检查除数是否为0,若有除0的操作输出nan。同时程序支持指数和负数运算。

\section{主要函数}
`parsenumber`函数检测字符类型并转化为相应类型。 `evaluator` 函数处理中缀表达式,并使用栈来计算结果。

\section{缺陷}
程序包含多个测试用例,包括合法和非法的输入,但是由于用的是catch和exit写的错误模块,没法一次性测试所有错误案例...另外由于写法比较简单所以速度上面可能有点欠缺。

\end{document}




